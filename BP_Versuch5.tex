\documentclass[paper=a4, fontsize=11pt]{scrartcl} 
\usepackage[utf8]{inputenc}
\usepackage{amsmath}
\usepackage{amsfonts}
\usepackage{amssymb}
\author{Kim Thuong Ngo}


\usepackage[T1]{fontenc} 
\usepackage{fourier} 

\usepackage{lipsum} 

\usepackage{listings}
\usepackage{graphicx}
\usepackage{tabularx}

\usepackage{sectsty}
\allsectionsfont{\centering \normalfont\scshape} 

\usepackage{fancyhdr} 
\pagestyle{fancyplain} 
\fancyhead{}
\fancyfoot[L]{} 
\fancyfoot[C]{} 
\fancyfoot[R]{\thepage} 
\renewcommand{\headrulewidth}{0pt} 
\renewcommand{\footrulewidth}{0pt}
\setlength{\headheight}{13.6pt}

\numberwithin{equation}{section} 
\numberwithin{figure}{section} 
\numberwithin{table}{section}

\setlength\parindent{0pt} 

\newcommand{\horrule}[1]{\rule{\linewidth}{#1}} 

\title{	
\normalfont \normalsize 
\textsc{Basispraktikum} \\ [25pt] 
\horrule{0.5pt} \\[0.4cm] 
\huge Ausarbeitung \\ 
\horrule{2pt} \\[0.5cm] 
}

\author{Jörg Gamerdinger und Kim Thuong Ngo} 

\date{\normalsize\today} 

%----------------------------------------------------------------------------------------

\begin{document}

\maketitle 

\newpage

\tableofcontents

\newpage

%----------------------------------------------------------------------------------------

\section{Rechenaufgaben}

%-----------------------------------------------------------------

\subsection{4+1}

\begin{tabular}{|c|c|cc|c|ccc|ccc|c|}
\hline
Adr. & Befehl &Steuerung & & Bus & Register & & & ALU & & & Flags \\
\hline
& & adr. ctrl. & next adr. & func. & adr. A & adr. B & write & in A & in B & funct. f= & load \\
\hline
0 & ADDS $R_{0}$ 4 & -  & 1 & - & 0 & 4 & A & - & C & ADDS & - \\
\hline
\end{tabular} \\

\begin{tabular}{|cc|cc|cccc|ccc|c|}
\hline
Steuerung & & Bus & & Register & & & & ALU & & & Flags \\
\hline
MAC & NA & WR & EN & AA & AB & WS & WE & Malu IA & Malu IB & Malus & MCH Flags \\
\hline
00 & 00001 & 0 & 0 & 000 & 0100 & 0 & 0 & 0 & 1 & 0101 & 0 \\
\hline
\end{tabular} \\

\underline{Erklärung}

Die Konstante $B=4$ wird als 0100 über MRG AB eingegeben. Mit dem Befehl ADDS, wird die Konstante mit dem Inhalt von Register $R_{0}$ und 1 addiert, d.h. folgende Rechnung $0+4+1$ wird ausgeführt. An den LEDs wird 0000 0101 ausgegeben, d.h. das Ergebnis ist 5 und die erwünschte Lösung.

%-----------------------------------------------------------------

\subsection{FF+1}

\begin{tabular}{|c|c|cc|c|ccc|ccc|c|}
\hline
Adr. & Befehl &Steuerung & & Bus & Register & & & ALU & & & Flags \\
\hline
& & adr. ctrl. & next adr. & func. & adr. A & adr. B & write & in A & in B & funct. f= & load \\
\hline
0 & ADDS $R_{0}$ FF & - & 0 & - & 0 & FF & - & - & C & ADDS & x  \\
\hline
\end{tabular} \\

\begin{tabular}{|cc|cc|cccc|ccc|c|}
\hline
Steuerung & & Bus & & Register & & & & ALU & & & Flags \\
\hline
MAC & NA & WR & EN & AA & AB & WS & WE & Malu IA & Malu IB & Malus & MCH Flags \\
\hline
00 & 00001 & 0 & 0 & 000 & 1111 & 0 & 0 & 0 & 1 & 0101 & 1\\
\hline
\end{tabular} \\

\underline{Erklärung}

Das Register FF wird als B eingelesen. Wie im vorigen Versuch a wird der Befehl ADDS verwendet und so $0 + FF+ 1$ addiert. Die LEDs geben als Ergebnis 0000 0000 an. Die Frage die aufkommt ist, warum es kein Carry Out gibt. Die Tabelle 5.1 zeigt an, dass der Befehl ADDS ein Carry besitzt, dieser aber invertiert ist. D.h. es gibt kein Carry Out bei dieser Rechnung.

%----------------------------------------------------------------------------------------

\newpage

\section{Mikroprogramme}

\begin{tabular}{|c|c|cc|c|ccc|ccc|c|}
\hline
Adr. & Befehl &Steuerung & & Bus & Register & & & ALU & & & Flags \\
\hline
& & adr. ctrl. & next adr. & func. & adr. A & adr. B & write & in A & in B & funct. f= & load \\
\hline
0 & LD $R_{4}$ FF & - & 1 & - & 4 & FF & A & - & C & B & - \\
\hline
1 & LD $R_{5}$ FE & - & 2 & - & 5 & FE & A & - & C & B & - \\
\hline
2 & LD $R_{1}$ ($R_{4}$) & - & 3 & rd & 4 & 1 & B & M & - & A & - \\
\hline
3 & LD $R_{2}$ ($R_{5}$) & - & 4 & rd & 5 & 2 & B & M & - & A & - \\
\hline
\end{tabular} \\

\begin{tabular}{|cc|cc|cccc|ccc|c|}
\hline
Steuerung & & Bus & & Register & & & & ALU & & & Flags \\
\hline
MAC & NA & WR & EN & AA & AB & WS & WE & Malu IA & Malu IB & Malus & MCH Flags \\
\hline
00 & 00001 & 0 & 0 & 100 & 1111 & 0 & 1 & 0 & 1 & 1100 & 0  \\
\hline
00 & 00010 & 0 & 0 & 101 & 1110 & 0 & 1 & 0 & 1 & 1100 & 0 \\
\hline
00 & 00011 & 0 & 1 & 100 & 0001 & 1 & 1 & 1 & 1 & 0001 & 0 \\
\hline
00 & 00100 & 0 & 1 & 101 & 0010 & 1 & 1 & 1 & 0 & 0001 & 0 \\
\hline
\end{tabular} \\

\underline{Erklärung}
Die Eingabe FF und FE werden als Adresse für Input-Register in den Registern $R_{4}$ und $R_{5}$ geladen. Mit dem Befehl LD wird auf die ALU die Werte aus F geliefert. Dieser ist am Dateneingang angekoppelt. Die Werte, aus den Memory Adressen FF und FE, werden in den Registern $R_{1}$ und $R_{2}$ geladen. Deshalb muss der Read Modus 0 sein und Bus Enable 1. \\

Die Adressen 0-3 dienen zur Initialisierung der Variablen A und B für die folgenden Teilaufgaben (a)-(c).

%-----------------------------------------------------------------

\subsection{(A+B)*2}

\begin{tabular}{|c|c|cc|c|ccc|ccc|c|}
\hline
Adr. & Befehl &Steuerung & & Bus & Register & & & ALU & & & Flags \\
\hline
& & adr. ctrl. & next adr. & func. & adr. A & adr. B & write & in A & in B & funct. f= & load \\
\hline
4 & ADD $R_{2}$ $R_{1}$ & - & 5 & - & 2 & 1 & B & R & R & ADD & - \\
\hline
5 & ADD $R_{1}$ $R_{1}$ & - & 6 & - & 1 & 1 & A & R & R & ADD & - \\
\hline
6 & LD $R_{3}$ FF & - & 7 & - & 3 & FF & A & - & C & B & - \\
\hline
7 & ST ($R_{3}$) $R_{1}$ & - & 0 & wr & 3 & 1 & - & - & R & B  & - \\
\hline
\end{tabular} \\

\begin{tabular}{|cc|cc|cccc|ccc|c|}
\hline
Steuerung & & Bus & & Register & & & & ALU & & & Flags \\
\hline
MAC & NA & WR & EN & AA & AB & WS & WE & Malu IA & Malu IB & Malus & MCH Flags \\
\hline
00 & 00101 & 0 & 0 & 010 & 0001 & 1 & 1 & 0 & 0 & 0100 & 0 \\
\hline
00 & 00110 & 0 & 0 & 001 & 0001 & 0 & 1 & 0 & 0 & 0100 & 0 \\
\hline
00 & 00111 & 0 & 0 & 011 & 1111 & 0 & 1 & 0 & 1 & 1100 & 0 \\
\hline
00 & 00000 & 1 & 1 & 011 & 0001 & 0 & 0 & 0 & 0 & 1100 & 0 \\
\hline
\end{tabular} \\

\underline{Erklärung}

Der Wert aus Register 1 wird mit dem Wert aus Register 2 addiert und das Ergebnis in Register 1 gespeichert ($A_{i+1}=A_{i}+B_{i}$). Somit ist die Summe aus A und B in Register 1 gespeichert. Die Summe soll mit 2 multipliziert werden, d.h. man addiert das Register mit sich selbst und speichert das Ergebnis wieder in Register 1 ($A_{i+2}=A_{i+1}+A_{i+1}$). Die Adresse vom Output FF wird in das Register 3 geladen. Mit dem letzten Befehl, wird das Ergebnis in Register 1 in das Output Register FF gespeichert. \\
D.h. das Ergebnis steht am Output FF. Nach Adresse 7 wird Adresse 0 angesteuert, das ist der Jump zum Anfang. Es schafft eine Endlosschleife, um das Programm immer durchlaufen lassen zu können.

%-----------------------------------------------------------------

\subsection{(A-B)/4}

\begin{tabular}{|c|c|cc|c|ccc|ccc|c|}
\hline
Adr. & Befehl &Steuerung & & Bus & Register & & & ALU & & & Flags \\
\hline
& & adr. ctrl. & next adr. & func. & adr. A & adr. B & write & in A & in B & funct. f= & load \\
\hline
\end{tabular} \\

\begin{tabular}{|cc|cc|cccc|ccc|c|}
\hline
Steuerung & & Bus & & Register & & & & ALU & & & Flags \\
\hline
MAC & NA & WR & EN & AA & AB & WS & WE & Malu IA & Malu IB & Malus & MCH Flags \\
\hline
\end{tabular} \\

%-----------------------------------------------------------------

\subsection{(A AND B)}

\begin{tabular}{|c|c|cc|c|ccc|ccc|c|}
\hline
Adr. & Befehl &Steuerung & & Bus & Register & & & ALU & & & Flags \\
\hline
& & adr. ctrl. & next adr. & func. & adr. A & adr. B & write & in A & in B & funct. f= & load \\
\hline
\end{tabular} \\

\begin{tabular}{|cc|cc|cccc|ccc|c|}
\hline
Steuerung & & Bus & & Register & & & & ALU & & & Flags \\
\hline
MAC & NA & WR & EN & AA & AB & WS & WE & Malu IA & Malu IB & Malus & MCH Flags \\
\hline
\end{tabular} \\

%----------------------------------------------------------------------------------------

\newpage

\section{Mikroprogramm zur Kontoführung}

\begin{tabular}{|c|c|cc|c|ccc|ccc|c|}
\hline
Adr. & Befehl &Steuerung & & Bus & Register & & & ALU & & & Flags \\
\hline
& & adr. ctrl. & next adr. & func. & adr. A & adr. B & write & in A & in B & funct. f= & load \\
\hline
\end{tabular} \\

\begin{tabular}{|cc|cc|cccc|ccc|c|}
\hline
Steuerung & & Bus & & Register & & & & ALU & & & Flags \\
\hline
MAC & NA & WR & EN & AA & AB & WS & WE & Malu IA & Malu IB & Malus & MCH Flags \\
\hline
\end{tabular} \\

%----------------------------------------------------------------------------------------

\newpage

\section{Versuchsprotokoll}

%----------------------------------------------------------------------------------------

\newpage

\section{Korrektur}


\end{document}