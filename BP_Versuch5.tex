\documentclass[paper=a4, fontsize=11pt]{scrartcl} 
\usepackage[utf8]{inputenc}
\usepackage{amsmath}
\usepackage{amsfonts}
\usepackage{amssymb}
\author{Kim Thuong Ngo}


\usepackage[T1]{fontenc} 
\usepackage{fourier} 

\usepackage{lipsum} 

\usepackage{listings}
\usepackage{graphicx}
\usepackage{tabularx}

\usepackage{sectsty}
\allsectionsfont{\centering \normalfont\scshape} 

\usepackage{fancyhdr} 
\pagestyle{fancyplain} 
\fancyhead{}
\fancyfoot[L]{} 
\fancyfoot[C]{} 
\fancyfoot[R]{\thepage} 
\renewcommand{\headrulewidth}{0pt} 
\renewcommand{\footrulewidth}{0pt}
\setlength{\headheight}{13.6pt}

\numberwithin{equation}{section} 
\numberwithin{figure}{section} 
\numberwithin{table}{section}

\setlength\parindent{0pt} 

\newcommand{\horrule}[1]{\rule{\linewidth}{#1}} 

\title{	
\normalfont \normalsize 
\textsc{Basispraktikum} \\ [25pt] 
\horrule{0.5pt} \\[0.4cm] 
\huge Ausarbeitung \\ 
\horrule{2pt} \\[0.5cm] 
}

\author{Jörg Gamerdinger und Kim Thuong Ngo} 

\date{\normalsize\today} 

%----------------------------------------------------------------------------------------

\begin{document}

\maketitle 

\newpage

\tableofcontents

\newpage

%----------------------------------------------------------------------------------------

\section{Rechenaufgaben}

%-----------------------------------------------------------------

\subsection{4+1}

\begin{tabular}{|c|c|cc|c|ccc|ccc|c|}
Adr. & Befehl &Steuerung & & Bus & Register & & & ALU & & Flags & \\
\hline
& & adr. ctrl. & next adr. & func. & adr. A & adr. B & write & in A & in B & funct. f= & load \\
\hline
\end{tabular}

\underline{Erklärung}

Die Konstante $B=4$ wird als 0100 über MRG AB eingegeben. Mit dem Befehl ADDS, wird die Konstante mit dem Inhalt von Register $R_{0}$ und 1 addiert, d.h. folgende Rechnung $0+4+1$ wird ausgeführt. An den LEDs wird 0000 0101 ausgegeben, d.h. das Ergebnis ist 5 und die erwünschte Lösung.

%-----------------------------------------------------------------

\subsection{FF+1}

%----------------------------------------------------------------------------------------

\newpage

\section{Mikroprogramme}

%-----------------------------------------------------------------

\subsection{(A+B)*2}

%-----------------------------------------------------------------

\subsection{(A-B)/4}

%-----------------------------------------------------------------

\subsection{(A AND B)}

%----------------------------------------------------------------------------------------

\newpage

\section{Mikroprogramm zur Kontoführung}

%----------------------------------------------------------------------------------------

\newpage

\section{Versuchsprotokoll}

%----------------------------------------------------------------------------------------

\newpage

\section{Korrektur}


\end{document}