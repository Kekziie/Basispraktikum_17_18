\documentclass[paper=a4, fontsize=11pt]{scrartcl} 
\usepackage[utf8]{inputenc}
\usepackage{amsmath}
\usepackage{amsfonts}
\usepackage{amssymb}
\author{Kim Thuong Ngo}


\usepackage[T1]{fontenc} 
\usepackage{fourier} 

\usepackage{lipsum} 

\usepackage{listings}
\usepackage{graphicx}
\usepackage{tabularx}

\usepackage{sectsty}
\allsectionsfont{\centering \normalfont\scshape} 

\usepackage{fancyhdr} 
\pagestyle{fancyplain} 
\fancyhead{}
\fancyfoot[L]{} 
\fancyfoot[C]{} 
\fancyfoot[R]{\thepage} 
\renewcommand{\headrulewidth}{0pt} 
\renewcommand{\footrulewidth}{0pt}
\setlength{\headheight}{13.6pt}

\numberwithin{equation}{section} 
\numberwithin{figure}{section} 
\numberwithin{table}{section}

\setlength\parindent{0pt} 

\newcommand{\horrule}[1]{\rule{\linewidth}{#1}} 

\title{	
\normalfont \normalsize 
\textsc{Basispraktikum} \\ [25pt] 
\horrule{0.5pt} \\[0.4cm] 
\huge Aufgaben \\ 
\horrule{2pt} \\[0.5cm] 
}

\author{Kim Thuong Ngo} 

\date{\normalsize\today} 

%----------------------------------------------------------------------------------------

\begin{document}

\maketitle 

\newpage

\tableofcontents

\newpage

%----------------------------------------------------------------------------------------

\section{Geräteeinführung}

\subsection{Geräteeinführung}

\subsection{Tiefpass}

\subsection{Kennlinie von Widerstand und Dioden}

%----------------------------------------------------------------------------------------

\newpage

\section{Digitale Elektronik}

\subsection{Kennlinie des n-Kanal-MOS-FET}

\subsection{Inverter und NOR-Gatter}

\subsection{XOR-Gatter}

\subsection{Inverterkette mit Rückkopplung}

\subsection{Hazards}

%----------------------------------------------------------------------------------------

\newpage

\section{Digitaler Schaltungsentwurf}

\subsection{Prellende Taster}

\subsection{Diskretes Master-Slave D-Flipflop}

\subsection{Synchroner 4-Bit Gray Zähler}

\subsection{1-Bit Addierer mit Carry In und Carry Out}

%----------------------------------------------------------------------------------------

\newpage

\section{Programmierbare Logik}

\subsection{Einführung}

\subsection{Register}

\subsection{4-Bit Multiplexer}

\subsection{4-Bit Addierer und Subtrahierer}

%----------------------------------------------------------------------------------------

\newpage

\section{Minirechner 2i}

\subsection{Bedienung des Minirechners}

\subsection{Rechenaufgaben}

\subsection{Mikroprogramme}

\subsection{Mikroprogramm zur Kontoführung}

%----------------------------------------------------------------------------------------

\newpage

\section{Minirechner 2a}

\subsection{einfache Maschinenprogramme}

\subsection{Assemblerprogramme}

\end{document}