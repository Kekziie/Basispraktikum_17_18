\documentclass[paper=a4, fontsize=11pt]{scrartcl} 
\usepackage[utf8]{inputenc}
\usepackage{amsmath}
\usepackage{amsfonts}
\usepackage{amssymb}
\author{Kim Thuong Ngo}


\usepackage[T1]{fontenc} 
\usepackage{fourier} 

\usepackage{lipsum} 

\usepackage{listings}
\usepackage{graphicx}
\usepackage{tabularx}

\usepackage{sectsty}
\allsectionsfont{\centering \normalfont\scshape} 

\usepackage{fancyhdr} 
\pagestyle{fancyplain} 
\fancyhead{}
\fancyfoot[L]{} 
\fancyfoot[C]{} 
\fancyfoot[R]{\thepage} 
\renewcommand{\headrulewidth}{0pt} 
\renewcommand{\footrulewidth}{0pt}
\setlength{\headheight}{13.6pt}

\numberwithin{equation}{section} 
\numberwithin{figure}{section} 
\numberwithin{table}{section}

\setlength\parindent{0pt} 

\newcommand{\horrule}[1]{\rule{\linewidth}{#1}} 

\title{	
\normalfont \normalsize 
\textsc{Basispraktikum} \\ [25pt] 
\horrule{0.5pt} \\[0.4cm] 
\huge Aufgaben \\ 
\horrule{2pt} \\[0.5cm] 
}

\author{Kim Thuong Ngo} 

\date{\normalsize\today} 

%----------------------------------------------------------------------------------------

\begin{document}

\maketitle 

\newpage

\tableofcontents

\newpage

%----------------------------------------------------------------------------------------

\section{Geräteeinführung}

\subsection{Geräteeinführung}

\begin{enumerate}
\item Überprüfen Sie ob sich die Tastköpfe in der Stellung 10x befinden.
\item Überprüfen Sie den Tastkopfabgleich für das Oszilloskop, indem Sie die Tastköpfe nacheinander mit dem im Oszilloskop eingebauten Signalgenerator verbinden. Es sollte eine saubere Rechtecksspannung ohne über- und untersteuernde Flanken dargestellt werden. Sollte dies nicht der Fall sein, so müssen die Tastköpfe abgeglichen werden. Wenden Sie sich dazu an einen Betreuer.
\item Erzeugen Sie mit Hilfe des Funktionsgenerators ein sinusförmiges Signal und legen Sie es auf einen der beiden Eingangskanäle des Oszilloskops. Verändern Sie nacheinander die Stellung der folgenden Regler: Horizontal Position, Vertical Position, Volts/Div, Sec/Div. Stellen Sie mit Hilfe der oben genannten Regler das Oszilloskop so ein, dass Sie die Frequenz des sinusförmigen Eingangssignal bestmöglich bestimmen können.
\item Skizzieren Sie maßstabsgerecht das Eingangssignal und bestimmen Sie rechnerisch dessen Frequenz! Stimmt diese mit der vom Oszilloskop berechneten Frequenz überein?
\end{enumerate}

\subsection{Tiefpass}

Bauen Sie auf dem analogen Experimentierboard einen Tiefpass mit der Schaltung aus Abbildung auf. Wählen Sie dazu den richtigen Widerstand aus und überprüfen Sie ihre Wahl mit Hilfe des Digitalmultimeters. \\

Verbinden Sie den Punkt A mit dem Ausgang des Funktionsgenerators. Dieser sollte auf 1k (= 1000 Hz) eingestellt sein. Messen Sie mit Kanal 1 des Oszilloskops den Spannungsverlauf U(A) am Punkt A, mit Kanal 2 den Spannungsverlauf U(B) an Punkt B. Verwenden Sie für den Funktiongenerator, das Oszilloskop und den Kondensator eine gemeinsame Referenz an Punkt R. Stellen Sie das Oszilloskop so ein, dass Sie die Eingangssignale optimal messen können. Achten Sie bei den folgenden Versuchen darauf, dass Sie für Kanal 1 und 2 am Oszilloskop immer die gleichen Einstellung wählen. \\

\begin{enumerate}
\item Berechnen Sie als Vorbereitung zuhause mit Hilfe einer geeigneten Formel die Grenzfrequenz der Schaltung. Welche Bedeutung hat diese Frequenz?
\item Messung des Tiefpasses für eine Sinusspannung
\begin{itemize}
\item Stellen Sie am Funktionsgenerator eine Sinusspannung von 1 kHz ein. Drehen Sie den Regler für die Frequenz über den gesamten Bereich. Beobachten und beschreiben Sie die Veränderungen.
\item Skizzieren Sie maßstabsgerecht die beiden gemessenen Kanäle bei einer Frequenz von 1000 Hz
\item Bestimmen Sie bei einer Frequenz von 1 kHz und 5 kHz das Spannungsverhältnis von Ausgangs- zu Eingangsspannung $\dfrac{U(B)}{U(A)}$ sowie die Phasenverschiebung.
\item Beschreiben Sie die Unterschiede der Kurven bei 1 kHz und 5 kHz.
\end{itemize}
\item Führen Sie die Aufgaben unter 2b ebenso für die Rechtecksspannung durch.
\item Machen Sie sich klar, wodurch die Phasenverschiebung bei einem Tiefpass wie in Abbildung entsteht und erklären Sie diese anhand Ihrer Beobachtungen bei angelegter Sinus- und Rechtecksspannung in Ihrer Ausarbeitung.
\end{enumerate}

\subsection{Kennlinie von Widerstand und Dioden}

Nehmen Sie die Kennlinien folgender Bauteile auf: ein $330 \Omega$ Widerstand, eine Universaldiode, eine rote und weiße Leuchtdiode. Verwenden Sie zur Messung folgende Schaltung. \\

Verbinden Sie den Punkt A mit dem Ausgang des Funktionsgenerators. Messen Sie mit Kanal 1 des Oszilloskops den Spannungsverlauf U(B) am Punkt B, mit Kanal 2 den Spannungsverlauf U(C) an Punkt C. Verwenden Sie für den Funktionsgenerator, das Oszilloskop und den $10 \Omega$ Messwiderstand eine gemeinsame Referenz an Punkt R. Stellen Sie das Oszilloskop und den Funktionsgenerator so ein, dass Sie die Eingangssignale im XY-Modus des Oszilloskops optimal messen können. \\

\begin{enumerate}

\item Vorbereitung
\begin{enumerate}
\item Machen Sie sich bei der Vorbereitung zuhause mit den allgemeinen Literaturkennlinien von Diode und Widerstand und ihren charakteristischen Eigenschaften vertraut und zeichnen Sie diese auf.
\item Wie funktioniert die Schaltung zur Messung der Kennlinie, welcher Messfehler entsteht dadurch?
\item Berechnen Sie den durch die verwendete Messschaltung bei der Messung von U(B) verursachten Messfehler bei $I = 6 mA$.
\item Wozu verwendet man den Funktionsgenerator; könnte man die Kennlinien auch ohne diesen erstellen?
\end{enumerate}

\item Messung
\begin{enumerate}
\item Skizzieren Sie jeweils maßstabsgetreu die Dioden-/Widerstandskennlinie.
\item Bestimmen Sie bei den Dioden jeweils die Durchlassspannung $U_{D}$ bei einem Durchlassstrom $I = 6 mA$ und zeichnen Sie bei den gemessenen Dioden $U_{D}$ in die Kennlinien ein.
\item Vergleichen Sie die gemessenen Kennlinien mit den allgemeinen Literaturkennlinien der Bauteile.
\end{enumerate}
\end{enumerate}

%----------------------------------------------------------------------------------------

\newpage

\section{Digitale Elektronik}

\subsection{Kennlinie des n-Kanal-MOS-FET}

\subsection{Inverter und NOR-Gatter}

\subsection{XOR-Gatter}

\subsection{Inverterkette mit Rückkopplung}

\subsection{Hazards}

%----------------------------------------------------------------------------------------

\newpage

\section{Digitaler Schaltungsentwurf}

\subsection{Prellende Taster}

\subsection{Diskretes Master-Slave D-Flipflop}

\subsection{Synchroner 4-Bit Gray Zähler}

\subsection{1-Bit Addierer mit Carry In und Carry Out}

%----------------------------------------------------------------------------------------

\newpage

\section{Programmierbare Logik}

\subsection{Einführung}

\subsection{Register}

\subsection{4-Bit Multiplexer}

\subsection{4-Bit Addierer und Subtrahierer}

%----------------------------------------------------------------------------------------

\newpage

\section{Minirechner 2i}

\subsection{Bedienung des Minirechners}

\subsection{Rechenaufgaben}

\subsection{Mikroprogramme}

\subsection{Mikroprogramm zur Kontoführung}

%----------------------------------------------------------------------------------------

\newpage

\section{Minirechner 2a}

\subsection{einfache Maschinenprogramme}

\subsection{Assemblerprogramme}

\end{document}