\documentclass[paper=a4, fontsize=11pt]{scrartcl} 
\usepackage[utf8]{inputenc}
\usepackage{amsmath}
\usepackage{amsfonts}
\usepackage{amssymb}
\author{Kim Thuong Ngo}


\usepackage[T1]{fontenc} 
\usepackage{fourier} 

\usepackage{lipsum} 

\usepackage{listings}
\usepackage{graphicx}
\usepackage{tabularx}

\usepackage{sectsty}
\allsectionsfont{\centering \normalfont\scshape} 

\usepackage{fancyhdr} 
\pagestyle{fancyplain} 
\fancyhead{}
\fancyfoot[L]{} 
\fancyfoot[C]{} 
\fancyfoot[R]{\thepage} 
\renewcommand{\headrulewidth}{0pt} 
\renewcommand{\footrulewidth}{0pt}
\setlength{\headheight}{13.6pt}

\numberwithin{equation}{section} 
\numberwithin{figure}{section} 
\numberwithin{table}{section}

\setlength\parindent{0pt} 

\newcommand{\horrule}[1]{\rule{\linewidth}{#1}} 

\title{	
\normalfont \normalsize 
\textsc{Basispraktikum} \\ [25pt] 
\horrule{0.5pt} \\[0.4cm] 
\huge Aufgaben \\ 
\horrule{2pt} \\[0.5cm] 
}

\author{Kim Thuong Ngo} 

\date{\normalsize\today} 

%----------------------------------------------------------------------------------------

\begin{document}

\maketitle 

\newpage

\tableofcontents

\newpage

%----------------------------------------------------------------------------------------

\section{Geräteeinführung}

\subsection{Geräteeinführung}

\begin{enumerate}
\item Überprüfen Sie ob sich die Tastköpfe in der Stellung 10x befinden.
\item Überprüfen Sie den Tastkopfabgleich für das Oszilloskop, indem Sie die Tastköpfe nacheinander mit dem im Oszilloskop eingebauten Signalgenerator verbinden. Es sollte eine saubere Rechtecksspannung ohne über- und untersteuernde Flanken dargestellt werden. Sollte dies nicht der Fall sein, so müssen die Tastköpfe abgeglichen werden. Wenden Sie sich dazu an einen Betreuer.
\item Erzeugen Sie mit Hilfe des Funktionsgenerators ein sinusförmiges Signal und legen Sie es auf einen der beiden Eingangskanäle des Oszilloskops. Verändern Sie nacheinander die Stellung der folgenden Regler: Horizontal Position, Vertical Position, Volts/Div, Sec/Div. Stellen Sie mit Hilfe der oben genannten Regler das Oszilloskop so ein, dass Sie die Frequenz des sinusförmigen Eingangssignal bestmöglich bestimmen können.
\item Skizzieren Sie maßstabsgerecht das Eingangssignal und bestimmen Sie rechnerisch dessen Frequenz! Stimmt diese mit der vom Oszilloskop berechneten Frequenz überein?
\end{enumerate}

\subsection{Tiefpass}

\subsection{Kennlinie von Widerstand und Dioden}

%----------------------------------------------------------------------------------------

\newpage

\section{Digitale Elektronik}

\subsection{Kennlinie des n-Kanal-MOS-FET}

\subsection{Inverter und NOR-Gatter}

\subsection{XOR-Gatter}

\subsection{Inverterkette mit Rückkopplung}

\subsection{Hazards}

%----------------------------------------------------------------------------------------

\newpage

\section{Digitaler Schaltungsentwurf}

\subsection{Prellende Taster}

\subsection{Diskretes Master-Slave D-Flipflop}

\subsection{Synchroner 4-Bit Gray Zähler}

\subsection{1-Bit Addierer mit Carry In und Carry Out}

%----------------------------------------------------------------------------------------

\newpage

\section{Programmierbare Logik}

\subsection{Einführung}

\subsection{Register}

\subsection{4-Bit Multiplexer}

\subsection{4-Bit Addierer und Subtrahierer}

%----------------------------------------------------------------------------------------

\newpage

\section{Minirechner 2i}

\subsection{Bedienung des Minirechners}

\subsection{Rechenaufgaben}

\subsection{Mikroprogramme}

\subsection{Mikroprogramm zur Kontoführung}

%----------------------------------------------------------------------------------------

\newpage

\section{Minirechner 2a}

\subsection{einfache Maschinenprogramme}

\subsection{Assemblerprogramme}

\end{document}