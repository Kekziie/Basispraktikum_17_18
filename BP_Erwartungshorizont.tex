\documentclass[paper=a4, fontsize=11pt]{scrartcl}
\usepackage[utf8]{inputenc}
\usepackage{amsmath}
\usepackage{amsfonts}
\usepackage{amssymb}
\author{Kim Thuong Ngo}


\usepackage[T1]{fontenc}
\usepackage{fourier}

\usepackage{lipsum}

\usepackage{listings}
\usepackage{graphicx}
\usepackage{tabularx}

\usepackage{sectsty}
\allsectionsfont{\centering \normalfont\scshape}

\usepackage{fancyhdr}
\pagestyle{fancyplain}
\fancyhead{}
\fancyfoot[L]{}
\fancyfoot[C]{}
\fancyfoot[R]{\thepage}
\renewcommand{\headrulewidth}{0pt}
\renewcommand{\footrulewidth}{0pt}
\setlength{\headheight}{13.6pt}

\numberwithin{equation}{section}
\numberwithin{figure}{section}
\numberwithin{table}{section}

\setlength\parindent{0pt}

\newcommand{\horrule}[1]{\rule{\linewidth}{#1}}

\title{
\normalfont \normalsize
\textsc{Basispraktikum} \\ [25pt]
\horrule{0.5pt} \\[0.4cm]
\huge Erwartungshorizont \\
\horrule{2pt} \\[0.5cm]
}

\author{Jörg Gamerdinger und Kim Thuong Ngo}

\date{\normalsize\today}

\begin{document}

\maketitle

\newpage

\tableofcontents

%----------------------------------------------------------------------------------------

\newpage

\section{Versuch 1: Geräteeinführung}

%-------------------------------------------------------------------

\subsection{Spannung, Potential und Strom}


%-------------------------------------------------------------------

\subsection{Geräte}


%-------------------------------------------------------------------

\subsection{Bauteile}


%-------------------------------------------------------------------

\subsection{Frequenz}


%-------------------------------------------------------------------

\subsection{Tiefpass}


%-------------------------------------------------------------------

\subsection{Kennlinien}


%----------------------------------------------------------------------------------------

\newpage

\section{Versuch 2: Digitale Elektronik}


%-------------------------------------------------------------------

\subsection{Bauteile}


%-------------------------------------------------------------------

\subsection{MOS-Transistor}


%-------------------------------------------------------------------

\subsection{CMOS-Technik}


%-------------------------------------------------------------------

\subsection{logische Gatter, Wahrheitstabellen und logische Verknüpfungen}


%-------------------------------------------------------------------

\subsection{Hazards}

%----------------------------------------------------------------------------------------

\newpage

\section{Versuch 3: Digitaler Schaltungsentwurf}

%-------------------------------------------------------------------

\subsection{Unterschied zwischen synchroner und asynchroner Schaltung}


%-------------------------------------------------------------------

\subsection{Flipflops}

%-------------------------------------------------------------------

\subsection{Prellen}

%-------------------------------------------------------------------

\subsection{Pull-Up Widerstand}

%-------------------------------------------------------------------

\subsection{systematischer Entwurf von Schaltwerken}

%-------------------------------------------------------------------

\subsection{Addierer}

%----------------------------------------------------------------------------------------

\newpage

\section{Versuch 4: Programmierbare Logik}

%-------------------------------------------------------------------

\subsection{PLD}

%-------------------------------------------------------------------

\subsection{Register}

%-------------------------------------------------------------------

\subsection{Multiplexer}

%----------------------------------------------------------------------------------------

\newpage

\section{Versuch 5: Minirechner 2i}

%-------------------------------------------------------------------

\subsection{Aufbau}

%-------------------------------------------------------------------

\subsection{Bedeutung der Bits}

%-------------------------------------------------------------------

\subsection{Ansteuerung und Funktionsweise der Adressen}

%-------------------------------------------------------------------

\subsection{Programmtabelle}

%----------------------------------------------------------------------------------------

\newpage

\section{Versuch 6: Minirechner 2a}

%-------------------------------------------------------------------

\subsection{Ansteuerung und Funktionsweise des Minirechners}

%-------------------------------------------------------------------

\subsection{Unterschiede der Minirechner 2i und 2a}

%-------------------------------------------------------------------

\subsection{Assembler}

%-------------------------------------------------------------------

\subsection{SP-Interface}

%-------------------------------------------------------------------

\subsection{Grundwissen zur Funktion der Bauteile}

%----------------------------------------------------------------------------------------

\end{document}
